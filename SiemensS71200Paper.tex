%%%%%%%%%%%%%%%%%%%%%%%%%%%%%%%%%%%%%%%%%%%%%%%%%%%%%%%%%%%%%%%%%%%%%%%%%%%%%%%%
% Template for USENIX papers.
%
% History:
%
% - TEMPLATE for Usenix papers, specifically to meet requirements of
%   USENIX '05. originally a template for producing IEEE-format
%   articles using LaTeX. written by Matthew Ward, CS Department,
%   Worcester Polytechnic Institute. adapted by David Beazley for his
%   excellent SWIG paper in Proceedings, Tcl 96. turned into a
%   smartass generic template by De Clarke, with thanks to both the
%   above pioneers. Use at your own risk. Complaints to /dev/null.
%   Make it two column with no page numbering, default is 10 point.
%
% - Munged by Fred Douglis <douglis@research.att.com> 10/97 to
%   separate the .sty file from the LaTeX source template, so that
%   people can more easily include the .sty file into an existing
%   document. Also changed to more closely follow the style guidelines
%   as represented by the Word sample file.
%
% - Note that since 2010, USENIX does not require endnotes. If you
%   want foot of page notes, don't include the endnotes package in the
%   usepackage command, below.
% - This version uses the latex2e styles, not the very ancient 2.09
%   stuff.
%
% - Updated July 2018: Text block size changed from 6.5" to 7"
%
% - Updated Dec 2018 for ATC'19:
%
%   * Revised text to pass HotCRP's auto-formatting check, with
%     hotcrp.settings.submission_form.body_font_size=10pt, and
%     hotcrp.settings.submission_form.line_height=12pt
%
%   * Switched from \endnote-s to \footnote-s to match Usenix's policy.
%
%   * \section* => \begin{abstract} ... \end{abstract}
%
%   * Make template self-contained in terms of bibtex entires, to allow
%     this file to be compiled. (And changing refs style to 'plain'.)
%
%   * Make template self-contained in terms of figures, to
%     allow this file to be compiled. 
%
%   * Added packages for hyperref, embedding fonts, and improving
%     appearance.
%   
%   * Removed outdated text.
%
%%%%%%%%%%%%%%%%%%%%%%%%%%%%%%%%%%%%%%%%%%%%%%%%%%%%%%%%%%%%%%%%%%%%%%%%%%%%%%%%

\documentclass[letterpaper,twocolumn,10pt]{article}
\usepackage{usenix-2020-09}

% to be able to draw some self-contained figs
\usepackage{tikz}
\usepackage{amsmath}

% inlined bib file
\usepackage{filecontents}

%-------------------------------------------------------------------------------
\begin{filecontents}{\jobname.bib}
%-------------------------------------------------------------------------------
@Book{RUB,
  author =       {RUB-SysSec},
  title =        {Siemens S7 PLCs Bootloader Arbitrary Code Execution Utility},
  publisher =    {Unbekannt},
  year =         2020,
  edition =      {1.00},
  note =         {\url{https://github.com/RUB-SysSec/SiemensS7-Bootloader}}
}
@InProceedings{beuth,
  author =       {Patrick Beuth},
  title =        {{"Stuxnet"-Entdeckung vor zehn Jahren|Die erste Cyberwaffe und ihre Folgen}},
  booktitle =    {{"Stuxnet"-Entdeckung vor zehn Jahren|Die erste Cyberwaffe und ihre Folgen}},
  year =         2020,
  pages =        {1--2},
  note =         {\url{https://www.spiegel.de/netzwelt/web/die-erste-cyberwaffe-und-ihre-folgen-a-a0ed08c9-5080-4ac2-8518-ed69347dc147}}}
\end{filecontents}

%-------------------------------------------------------------------------------
\begin{document}
%-------------------------------------------------------------------------------

%don't want date printed
\date{}

% make title bold and 14 pt font (Latex default is non-bold, 16 pt)
\title{\Large \bf Sicherheit industrieller Steuerungssysteme:\\
  Siemens Simatic-S7 1200}

%for single author (just remove % characters)
\author{
{\rm  Ch.\ Weis}\\
%\and
%{\rm Second Name}\\
%Second Institution
% copy the following lines to add more authors
% \and
% {\rm Name}\\
%Name Institution
} % end author

\maketitle

%-------------------------------------------------------------------------------
\begin{abstract}
%-------------------------------------------------------------------------------
Das Projekt beschäftigt sich mit der Sicherheit und verschiedenen Angriffsvektoren eines Simatic-S7 Steuerungssystem.
Verwendet wurde ein Gerät der 1200er Reihe.
\end{abstract}


%-------------------------------------------------------------------------------
\section{Einleitung}
%-------------------------------------------------------------------------------

Heutzutage sind intelligente Steuerungssysteme von Nöten um in Industrieanlagen effizient zu produzieren.
Sicherheit spielt dabei immer eine wichtige Rolle wie man zum Beispiel beim Virus Stuxnet erkennt.
Ins besonders sind die oben genannten Siemens Systeme gefährdet, da diese am meisten verwendet werden.

%-------------------------------------------------------------------------------
\section{Werkzeuge und Materialien}
%-------------------------------------------------------------------------------

Verwendet wurden die Tools Nmap und Wireshark.\\
\\
\noindent{\large Nmap:} Nmap, von Gordon Lyon entwickelt, ist ein Tool um Netzwerke oder auch einzelne IP-Adressen zu scannen, um Services und Ports zu finden. 
Nmap zeigt die offenen Ports mit den jeweiligen Protokollen an.
	  
\noindent{\large Wireshark:} Wireshark, von der Wireshark-Community entwickelt, wird verwendet um Netzwerkverbindung zu überwachen und mitzuschneiden.
 Hiermit kann man IP-Adressen, MAC-Adressen und weitere Informationen aus einem Netzwerk mitschneiden und abhören.

\noindent{\large S7-1200:} Der Siemens Simatic-S7 1200 mit 1212C AC/DC/RLY CPU wurde als Ziel-Gerät ausgewählt, da er ein häufig verwendetes Gerät der Industrie ist. 
Das S7-System wird zu Automation von großen Industrieanlagen verwendet, indem es selber auf Daten von Sensoren reagiert, wenn es vorher einprogrammiert wurde, reagiert.
Das System kann über verschiedenen Programme gesteuert werden.



\cite{RUB}
\cite{beuth}
%~\cite{waldspurger02,arpachiDusseau18:osbook}. 

%~\ref{sec:figs}, but not before


%-------------------------------------------------------------------------------
\section{Vorgehen}
\label{sec:figs}
%-------------------------------------------------------------------------------
Als erstes musste man das Gerät mit Nmap scannen um herauszufinden welche Protokolle von dem Gerät gesprochen werden.
Dafür wurde der "-sS"-Scan verwendet und mit dem "-sV"-Scan ergänzt.
"-sS" scannt das Ziel-Gerät im "stealty"-Modus und "-sV" scannt nach der Version verwendeter Services.
Die Scans wurden mit Wireshark mitgeschnitten und überwacht. Man konnte dann beide offenen Ports sehen und das Protokoll sich anzeigen lassen.

\begin{verbatim}
	PORT 		STATE 	SERVICE
	102/tcp 	open	ios-tsap
	4840/tcp 	open	opcua
\end{verbatim}




%---------------------------
%\begin{figure}
%\begin{center}
%\begin{tikzpicture}
 % \draw[thin,gray!40] (-2,-2) grid (2,2);
  %\draw[<->] (-2,0)--(2,0) node[right]{$x$};
  %\draw[<->] (0,-2)--(0,2) node[above]{$y$};
  %\draw[line width=2pt,blue,-stealth](0,0)--(1,1)
   %     node[anchor=south west]{$\boldsymbol{u}$};
  %\draw[line width=2pt,red,-stealth](0,0)--(-1,-1)
   %     node[anchor=north east]{$\boldsymbol{-u}$};
%\end{tikzpicture}
%\end{center}
%\caption{\label{fig:vectors} Text size inside figure should be as big as
 % caption's text. Text size inside figure should be as big as
  %caption's text. Text size inside figure should be as big as
  %caption's text. Text size inside figure should be as big as
  %caption's text. Text size inside figure should be as big as
  %caption's text. }
%\end{figure}
%% %---------------------------


\noindent Nun wurden Implementierungen für die verschiedenen Protokolle gesucht.
Gefunden wurde bei der Recherche nur ein Verweise auf eine Schwachstelle(CVE-2019-13945) mit dem Protokoll opcua, welches in dem Gerät vorhanden seien soll.
Die Schwachstelle befindet sich im Bootloader und kann wärend des Bootvorgangs in einem Zeitfenster von genau einer halbe Sekunde 
exploitet werden. Der Bootloader muss die Version 4.2.1 vorweisen. Wärend des Zeitfensters könnte man dann ein Payload aufspielen und es danach ausführen. 
Auch außerhalb des Zeitfensters. 
Um eine Verbindung mit dem S7 1200 zu bilden muss der Computer über ein Adapter für serielle Verbindungen verfügen.
Auf dem mir zugänglichen Gerät aber befindet sich eine neuere Bootloader Version, in der der Exploit nicht mehr vorhanden ist.
Auf verschiedenen Websites ist kein Verweis auf das Erscheinungsdatum verschiedener Bootloaderversionen. Einzig die Firmware wird aufgeführt, 
jedoch ohne Erscheinungsdaten.\\
Darauf wurde versucht das Siemens Simatic-Step7 System auf einem Windowsrechner zu installieren. 
Jedoch ohne Erfolg.


%\begin{description}
  
%\item[fread] a function that reads from a \texttt{stream} into the
%  array \texttt{ptr} at most \texttt{nobj} objects of size
%  \texttt{size}, returning returns the number of objects read.

%\item[Fred] a person's name, e.g., there once was a dude named Fred
%  who separated usenix.sty from this file to allow for easy
%  inclusion.
%\end{description}


\subsection{Schlussfolgerung}
%-----------------------------------

Die oben genannte Schwachstelle(CVE-2019-13945) konnte nicht nachgewiesen werden und trotzdem ist diese Schwachstelle nicht zu unterschätzen, da doch große Schäden bei älteren Geräten auftreten können, wenn diese Schwachstelle von Angreifern gefunden wird.

%-------------------------------------------------------------------------------
%\section*{Acknowledgments}
%-------------------------------------------------------------------------------

%The USENIX latex style is old and very tired, which is why
%there's no \textbackslash{}acks command for you to use when
%acknowledging. Sorry.

%-------------------------------------------------------------------------------
%\section*{Availability}
%-------------------------------------------------------------------------------

%USENIX program committees give extra points to submissions that are
%backed by artifacts that are publicly available. If you made your code
%or data available, it's worth mentioning this fact in a dedicated
%section.

%-------------------------------------------------------------------------------
\bibliographystyle{plain}
\bibliography{\jobname}

%%%%%%%%%%%%%%%%%%%%%%%%%%%%%%%%%%%%%%%%%%%%%%%%%%%%%%%%%%%%%%%%%%%%%%%%%%%%%%%%
\end{document}
%%%%%%%%%%%%%%%%%%%%%%%%%%%%%%%%%%%%%%%%%%%%%%%%%%%%%%%%%%%%%%%%%%%%%%%%%%%%%%%%

%%  LocalWords:  endnotes includegraphics fread ptr nobj noindent
%%  LocalWords:  pdflatex acks
